\documentclass{article}
\begin{document}

\ifpdf%                                % if we use pdflatex
  \pdfoutput=1\relax                   % create PDFs from pdfLaTeX
  \pdfcompresslevel=9                  % PDF Compression
  \pdfoptionpdfminorversion=7          % create PDF 1.7
  \ExecuteOptions{pdftex}
  \usepackage{graphicx}                % allow us to embed graphics files
  \DeclareGraphicsExtensions{.pdf,.png,.jpg,.jpeg} % for pdflatex we expect .pdf, .png, or .jpg files
\else%                                 % else we use pure latex
  \ExecuteOptions{dvips}
  \usepackage{graphicx}                % allow us to embed graphics files
  \DeclareGraphicsExtensions{.eps}     % for pure latex we expect eps files
\fi%

%% it is recomended to use ``\autoref{sec:bla}'' instead of ``Fig.~\ref{sec:bla}''
\graphicspath{{figures/}{pictures/}{images/}{./}} % where to search for the images

\usepackage{microtype}                 % use micro-typography (slightly more compact, better to read)
\PassOptionsToPackage{warn}{textcomp}  % to address font issues with \textrightarrow
\usepackage{textcomp}                  % use better special symbols
\usepackage{mathptmx}                  % use matching math font
\usepackage{times}                     % we use Times as the main font
\renewcommand*\ttdefault{txtt}         % a nicer typewriter font
\usepackage{cite}                      % needed to automatically sort the references
\usepackage{tabu}                      % only used for the table example
\usepackage{booktabs}                  % only used for the table example

%% We encourage the use of mathptmx for consistent usage of times font
%% throughout the proceedings. However, if you encounter conflicts
%% with other math-related packages, you may want to disable it.

%% In preprint mode you may define your own headline.
%\preprinttext{To appear in IEEE Transactions on Visualization and Computer Graphics.}

%% If you are submitting a paper to a conference for review with a double
%% blind reviewing process, please replace the value ``0'' below with your
%% OnlineID. Otherwise, you may safely leave it at ``0''.
\onlineid{0}

%% declare the category of your paper, only shown in review mode
\vgtccategory{Research}
%% please declare the paper type of your paper to help reviewers, only shown in review mode
%% choices:
%% * algorithm/technique
%% * application/design study
%% * evaluation
%% * system
%% * theory/model
\vgtcpapertype{please specify}

%% Paper title.
\title{Global Illumination for Fun and Profit}

%% This is how authors are specified in the journal style

%% indicate IEEE Member or Student Member in form indicated below
\author{Roy G. Biv, Ed Grimley, \textit{Member, IEEE}, and Martha Stewart}
\authorfooter{
%% insert punctuation at end of each item
\item
 Roy G. Biv is with Starbucks Research. E-mail: roy.g.biv@aol.com.
\item
 Ed Grimley is with Grimley Widgets, Inc.. E-mail: ed.grimley@aol.com.
\item
 Martha Stewart is with Martha Stewart Enterprises at Microsoft
 Research. E-mail: martha.stewart@marthastewart.com.
}

%other entries to be set up for journal
\shortauthortitle{Biv \MakeLowercase{\textit{et al.}}: Global Illumination for Fun and Profit}
%\shortauthortitle{Firstauthor \MakeLowercase{\textit{et al.}}: Paper Title}

%% Abstract section.
\abstract{Duis autem vel eum iriure dolor in hendrerit in vulputate
velit esse molestie consequat, vel illum dolore eu feugiat nulla
facilisis at vero eros et accumsan et iusto odio dignissim qui blandit
praesent luptatum zzril delenit augue duis dolore te feugait nulla
facilisi. Lorem ipsum dolor sit amet, consectetuer adipiscing elit,
sed diam nonummy nibh euismod tincidunt ut laoreet dolore magna
aliquam erat volutpat. Ut wisi enim ad minim veniam, quis nostrud exerci tation ullamcorper
suscipit lobortis nisl ut aliquip ex ea commodo consequat. Duis autem
vel eum iriure dolor in hendrerit in vulputate velit esse molestie
consequat, vel illum dolore eu feugiat nulla facilisis at vero eros et
accumsan et iusto odio dignissim qui blandit praesent luptatum zzril
delenit augue duis dolore te feugait nulla facilisi.%
} % end of abstract

%% Keywords that describe your work. Will show as 'Index Terms' in journal
%% please capitalize first letter and insert punctuation after last keyword
\keywords{Radiosity, global illumination, constant time}

%% ACM Computing Classification System (CCS). 
%% See <http://www.acm.org/class/1998/> for details.
%% The ``\CCScat'' command takes four arguments.

\CCScatlist{ % not used in journal version
 \CCScat{K.6.1}{Management of Computing and Information Systems}%
{Project and People Management}{Life Cycle};
 \CCScat{K.7.m}{The Computing Profession}{Miscellaneous}{Ethics}
}

%% Uncomment below to include a teaser figure.
\teaser{
  \centering
  \includegraphics[width=\linewidth]{CypressView}
  \caption{In the Clouds: Vancouver from Cypress Mountain. Note that the teaser may not be wider than the abstract block.}
	\label{fig:teaser}
}


\vgtcinsertpkg

%%%%%%%%%%%%%%%%%%%%%%%%%%%%%%%%%%%%%%%%%%%%%%%%%%%%%%%%%%%%%%%%
%%%%%%%%%%%%%%%%%%%%%% START OF THE PAPER %%%%%%%%%%%%%%%%%%%%%%
%%%%%%%%%%%%%%%%%%%%%%%%%%%%%%%%%%%%%%%%%%%%%%%%%%%%%%%%%%%%%%%%%

\begin{document}

%% The ``\maketitle'' command must be the first command after the
%% ``\begin{document}'' command. It prepares and prints the title block.

%% the only exception to this rule is the \firstsection command
\firstsection{Introduction}

\maketitle

%% \section{Introduction} %for journal use above \firstsection{..} instead
This template is for papers of VGTC-sponsored conferences such as IEEE VIS, IEEE VR, and ISMAR which are published as special issues of TVCG. The template does not contain the respective dates of the conference/journal issue, these will be entered by IEEE as part of the publication production process. Therefore, \textbf{please leave the copyright statement at the bottom-left of this first page untouched}.

\section{Using the Style Template}

\begin{itemize}
\item If you receive compilation errors along the lines of ``\texttt{Package ifpdf Error: Name clash, \textbackslash ifpdf is already defined}'' then please add a new line ``\texttt{\textbackslash let\textbackslash ifpdf\textbackslash relax}'' right after the ``\texttt{\textbackslash documentclass[journal]\{vgtc\}}'' call. Note that your error is due to packages you use that define ``\texttt{\textbackslash ifpdf}'' which is obsolete (the result is that \texttt{\textbackslash ifpdf} is defined twice); these packages should be changed to use ifpdf package instead.
\item Note that each author's affiliations have to be provided in the author footer on the bottom-left corner of the first page. It is permitted to merge two or more people from the same institution as long as they are shown in the same order as in the overall author sequence on the top of the first page. For example, if authors A, B, C, and D are from institutions 1, 2, 1, and 2, respectively, then it is ok to use 2 bullets as follows:
\begin{itemize}
\item A and C are with Institution 1. E-mail: \{a\,$|$\,c\}@i1.com\,.
\item B and D are with Institution 2. E-mail: \{b\,$|$\,d\}@i2.org\,.
\end{itemize}
\item The style uses the hyperref package, thus turns references into internal links. We thus recommend to make use of the ``\texttt{\textbackslash autoref\{reference\}}'' call (instead of ``\texttt{Figure\~{}\textbackslash ref\{reference\}}'' or similar) since ``\texttt{\textbackslash autoref\{reference\}}'' turns the entire reference into an internal link, not just the number. Examples: \autoref{fig:sample} and \autoref{tab:vis_papers}.
\item The style automatically looks for image files with the correct extension (eps for regular \LaTeX; pdf, png, and jpg for pdf\LaTeX), in a set of given subfolders (figures/, pictures/, images/). It is thus sufficient to use ``\texttt{\textbackslash includegraphics\{CypressView\}}'' (instead of ``\texttt{\textbackslash includegraphics\{pictures/CypressView.jpg\}}'').
\item For adding hyperlinks and DOIs to the list of references, you can use ``\texttt{\textbackslash bibliographystyle\{abbrv-doi-hyperref-narrow\}}'' (instead of ``\texttt{\textbackslash bibliographystyle\{abbrv\}}''). It uses the doi and url fields in a bib\TeX\ entry and turns the entire reference into a link, giving priority to the doi. The doi can be entered with or without the ``\texttt{http://dx.doi.org/}'' url part. See the examples in the bib\TeX\ file and the bibliography at the end of this template.\\[1em]
\textbf{Note 1:} occasionally (for some \LaTeX\ distributions) this hyper-linked bib\TeX\ style may lead to \textbf{compilation errors} (``\texttt{pdfendlink ended up in different nesting level ...}'') if a reference entry is broken across two pages (due to a bug in hyperref). In this case make sure you have the latest version of the hyperref package (i.\,e., update your \LaTeX\ installation/packages) or, alternatively, revert back to ``\texttt{\textbackslash bibliographystyle\{abbrv-doi-narrow\}}'' (at the expense of removing hyperlinks from the bibliography) and try ``\texttt{\textbackslash bibliographystyle\{abbrv-doi-hyperref-narrow\}}'' again after some more editing.\\[1em]
\textbf{Note 2:} the ``\texttt{-narrow}'' versions of the bibliography style use the font ``PTSansNarrow-TLF'' for typesetting the DOIs in a compact way. This font needs to be available on your \LaTeX\ system. It is part of the \href{https://www.ctan.org/pkg/paratype}{``paratype'' package}, and many distributions (such as MikTeX) have it automatically installed. If you do not have this package yet and want to use a ``\texttt{-narrow}'' bibliography style then use your \LaTeX\ system's package installer to add it. If this is not possible you can also revert to the respective bibliography styles without the ``\texttt{-narrow}'' in the file name.\\[1em]
DVI-based processes to compile the template apparently cannot handle the different font so, by default, the template file uses the \texttt{abbrv-doi} bibliography style but the compiled PDF shows you the effect of the \texttt{abbrv-doi-hyperref-narrow} style.
\end{itemize}

\section{Bibliography Instructions}

\begin{itemize}
\item Sort all bibliographic entries alphabetically but the last name of the first author. This \LaTeX/bib\TeX\ template takes care of this sorting automatically.
\item Merge multiple references into one; e.\,g., use \cite{Max:1995:OMF,Kitware:2003} (not \cite{Kitware:2003}\cite{Max:1995:OMF}). Within each set of multiple references, the references should be sorted in ascending order. This \LaTeX/bib\TeX\ template takes care of both the merging and the sorting automatically.
\item Verify all data obtained from digital libraries, even ACM's DL and IEEE Xplore  etc.\ are sometimes wrong or incomplete.
\item Do not trust bibliographic data from other services such as Mendeley.com, Google Scholar, or similar; these are even more likely to be incorrect or incomplete.
\item Articles in journal---items to include:
  \begin{itemize}
  \item author names
	\item title
	\item journal name
	\item year
	\item volume
	\item number
	\item month of publication as variable name (i.\,e., \{jan\} for January, etc.; month ranges using \{jan \#\{/\}\# feb\} or \{jan \#\{-{}-\}\# feb\})
  \end{itemize}
\item use journal names in proper style: correct: ``IEEE Transactions on Visualization and Computer Graphics'', incorrect: ``Visualization and Computer Graphics, IEEE Transactions on''
\item Papers in proceedings---items to include:
  \begin{itemize}
  \item author names
	\item title
	\item abbreviated proceedings name: e.\,g., ``Proc.\textbackslash{} CONF\_ACRONYNM'' without the year; example: ``Proc.\textbackslash{} CHI'', ``Proc.\textbackslash{} 3DUI'', ``Proc.\textbackslash{} Eurographics'', ``Proc.\textbackslash{} EuroVis''
	\item year
	\item publisher
	\item town with country of publisher (the town can be abbreviated for well-known towns such as New York or Berlin)
  \end{itemize}
\item article/paper title convention: refrain from using curly brackets, except for acronyms/proper names/words following dashes/question marks etc.; example:
\begin{itemize}
	\item paper ``Marching Cubes: A High Resolution 3D Surface Construction Algorithm''
	\item should be entered as ``\{M\}arching \{C\}ubes: A High Resolution \{3D\} Surface Construction Algorithm'' or  ``\{M\}arching \{C\}ubes: A high resolution \{3D\} surface construction algorithm''
	\item will be typeset as ``Marching Cubes: A high resolution 3D surface construction algorithm''
\end{itemize}
\item for all entries
\begin{itemize}
	\item DOI can be entered in the DOI field as plain DOI number or as DOI url; alternative: a url in the URL field
	\item provide full page ranges AA-{}-BB
\end{itemize}
\item when citing references, do not use the reference as a sentence object; e.\,g., wrong: ``In \cite{Lorensen:1987:MCA} the authors describe \dots'', correct: ``Lorensen and Cline \cite{Lorensen:1987:MCA} describe \dots''
\end{itemize}

\section{Example Section}

Lorem\marginpar{\small You can use the margins for comments while editing the submission, but please remove the marginpar comments for submission.} ipsum dolor sit amet, consetetur sadipscing elitr, sed diam
nonumy eirmod tempor invidunt ut labore et dolore magna aliquyam erat,
sed diam voluptua. At vero eos et accusam et justo duo dolores et ea
rebum. Stet clita kasd gubergren, no sea takimata sanctus est Lorem
ipsum dolor sit amet. Lorem ipsum dolor sit amet, consetetur
sadipscing elitr, sed diam nonumy eirmod tempor invidunt ut labore et
dolore magna aliquyam erat, sed diam
voluptua~\cite{Kitware:2003,Max:1995:OMF}. At vero eos et accusam et
justo duo dolores et ea rebum. Stet clita kasd gubergren, no sea
takimata sanctus est Lorem ipsum dolor sit amet. Lorem ipsum dolor sit
amet, consetetur sadipscing elitr, sed diam nonumy eirmod tempor
invidunt ut labore et dolore magna aliquyam erat, sed diam
voluptua. At vero eos et accusam et justo duo dolores et ea
rebum. Stet clita kasd gubergren, no sea takimata sanctus est.

\section{Exposition}

Duis autem vel eum iriure dolor in hendrerit in vulputate velit esse
molestie consequat, vel illum dolore eu feugiat nulla facilisis at
vero eros et accumsan et iusto odio dignissim qui blandit praesent
luptatum zzril delenit augue duis dolore te feugait nulla
facilisi. Lorem ipsum dolor sit amet, consectetuer adipiscing elit,
sed diam nonummy nibh euismod tincidunt ut laoreet dolore magna
aliquam erat volutpat~\cite{Kindlmann:1999:SAG}.

\begin{equation}
\sum_{j=1}^{z} j = \frac{z(z+1)}{2}
\end{equation}

Lorem ipsum dolor sit amet, consetetur sadipscing elitr, sed diam
nonumy eirmod tempor invidunt ut labore et dolore magna aliquyam erat,
sed diam voluptua. At vero eos et accusam et justo duo dolores et ea
rebum. Stet clita kasd gubergren, no sea takimata sanctus est Lorem
ipsum dolor sit amet. Lorem ipsum dolor sit amet, consetetur
sadipscing elitr, sed diam nonumy eirmod tempor invidunt ut labore et
dolore magna aliquyam erat, sed diam voluptua. At vero eos et accusam
et justo duo dolores et ea rebum. Stet clita kasd gubergren, no sea
takimata sanctus est Lorem ipsum dolor sit amet.

\subsection{Lorem ipsum}

Lorem ipsum dolor sit amet (see \autoref{tab:vis_papers}), consetetur sadipscing elitr, sed diam
nonumy eirmod tempor invidunt ut labore et dolore magna aliquyam erat,
sed diam voluptua. At vero eos et accusam et justo duo dolores et ea
rebum. Stet clita kasd gubergren, no sea takimata sanctus est Lorem
ipsum dolor sit amet. Lorem ipsum dolor sit amet, consetetur
sadipscing elitr, sed diam nonumy eirmod tempor invidunt ut labore et
dolore magna aliquyam erat, sed diam voluptua. At vero eos et accusam
et justo duo dolores et ea rebum. Stet clita kasd gubergren, no sea
takimata sanctus est Lorem ipsum dolor sit amet. Lorem ipsum dolor sit
amet, consetetur sadipscing elitr, sed diam nonumy eirmod tempor
invidunt ut labore et dolore magna aliquyam erat, sed diam
voluptua. At vero eos et accusam et justo duo dolores et ea
rebum. 


\begin{table}[tb]
  \caption{VIS/VisWeek accepted/presented papers: 1990--2016.}
  \label{tab:vis_papers}
  \scriptsize%
	\centering%
  \begin{tabu}{%
	r%
	*{7}{c}%
	*{2}{r}%
	}
  \toprule
   year & \rotatebox{90}{Vis/SciVis} &   \rotatebox{90}{SciVis conf} &   \rotatebox{90}{InfoVis} &   \rotatebox{90}{VAST} &   \rotatebox{90}{VAST conf} &   \rotatebox{90}{TVCG @ VIS} &   \rotatebox{90}{CG\&A @ VIS} &   \rotatebox{90}{VIS/VisWeek} \rotatebox{90}{incl. TVCG/CG\&A}   &   \rotatebox{90}{VIS/VisWeek} \rotatebox{90}{w/o TVCG/CG\&A}   \\
  \midrule
	2016 & 30 &   & 37 & 33 & 15 & 23 & 10 & 148 & 115 \\
  2015 & 33 & 9 & 38 & 33 & 14 & 17 & 15 & 159 & 127 \\
  2014 & 34 &   & 45 & 33 & 21 & 20 &   & 153 & 133 \\
  2013 & 31 &   & 38 & 32 &   & 20 &   & 121 & 101 \\
  2012 & 42 &   & 44 & 30 &   & 23 &   & 139 & 116 \\
  2011 & 49 &   & 44 & 26 &   & 20 &   & 139 & 119 \\
  2010 & 48 &   & 35 & 26 &   &   &   & 109 & 109 \\
  2009 & 54 &   & 37 & 26 &   &   &   & 117 & 117 \\
  2008 & 50 &   & 28 & 21 &   &   &   & 99 & 99 \\
  2007 & 56 &   & 27 & 24 &   &   &   & 107 & 107 \\
  2006 & 63 &   & 24 & 26 &   &   &   & 113 & 113 \\
  2005 & 88 &   & 31 &   &   &   &   & 119 & 119 \\
  2004 & 70 &   & 27 &   &   &   &   & 97 & 97 \\
  2003 & 74 &   & 29 &   &   &   &   & 103 & 103 \\
  2002 & 78 &   & 23 &   &   &   &   & 101 & 101 \\
  2001 & 74 &   & 22 &   &   &   &   & 96 & 96 \\
  2000 & 73 &   & 20 &   &   &   &   & 93 & 93 \\
  1999 & 69 &   & 19 &   &   &   &   & 88 & 88 \\
  1998 & 72 &   & 18 &   &   &   &   & 90 & 90 \\
  1997 & 72 &   & 16 &   &   &   &   & 88 & 88 \\
  1996 & 65 &   & 12 &   &   &   &   & 77 & 77 \\
  1995 & 56 &   & 18 &   &   &   &   & 74 & 74 \\
  1994 & 53 &   &   &   &   &   &   & 53 & 53 \\
  1993 & 55 &   &   &   &   &   &   & 55 & 55 \\
  1992 & 53 &   &   &   &   &   &   & 53 & 53 \\
  1991 & 50 &   &   &   &   &   &   & 50 & 50 \\
  1990 & 53 &   &   &   &   &   &   & 53 & 53 \\
  \midrule
  \textbf{sum} & \textbf{1545} & \textbf{9} & \textbf{632} & \textbf{310} & \textbf{50} & \textbf{123} & \textbf{25} & \textbf{2694} & \textbf{2546} \\
  \bottomrule
  \end{tabu}%
\end{table}

\end{document}
